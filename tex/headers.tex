\usepackage{amsmath,amsfonts,amssymb,tikz,multirow,stmaryrd,xspace}
\usepackage[hyphens]{url}
\usepackage{varwidth}
\usepackage{amsthm}
\usepackage{caption}

\usepackage{amssymb}
\usepackage{pgfplots}
\pgfplotsset{compat=newest}

%this should save a minimum amount of space and in theory look nicer
\usepackage{ellipsis, ragged2e, marginnote}
\usepackage[tracking=true]{microtype}
\DeclareMicrotypeSet*[tracking]{my}{ font = */*/*/sc/* }
\SetTracking{encoding = *, shape = sc}{45}

\usepackage[]{hyperref}

\hypersetup{
    colorlinks=true,
    linkcolor=darkred,
    citecolor=darkgreen,
    urlcolor=darkblue
}

\newcommand{\false}{\ensuremath{\textsc{false}}\xspace}
\newcommand{\true}{\ensuremath{\textsc{true}}\xspace}

\newcommand{\naive}{na\"{\i}ve\xspace}
\newcommand{\Naive}{Na\"{\i}ve\xspace}

\newcommand{\share}[1]{\ensuremath{\llbracket #1\rrbracket}\xspace}
\newcommand{\F}{\ensuremath{\mathbb{F}}\xspace}
\newcommand{\C}{\ensuremath{\mathcal{C}}\xspace}
\newcommand{\M}{\ensuremath{\mathcal{M}}\xspace}
\renewcommand{\O}{\ensuremath{\mathcal{O}}\xspace}
\renewcommand{\P}{\ensuremath{\mathcal{P}}\xspace}
\newcommand{\A}{\ensuremath{\mathcal{A}}\xspace}
\newcommand{\R}{\ensuremath{\mathcal{R}}\xspace}

\newcommand{\myvec}[1]{\boldsymbol{#1}}

\newcommand{\algorithm}[1]{\ensuremath{\text{\sf #1}}\xspace}

\newcommand{\DPFgen}{\algorithm{DPF.Gen}}
\newcommand{\DPFexpand}{\algorithm{DPF.Expand}}

\newcommand{\serv}[1]{server~\##1}
\newcommand{\Serv}[1]{Server~\##1}
\def\SSS{\ensuremath{\mathcal{S}}\xspace}
\def\RR{\ensuremath{\mathcal{R}}\xspace}
\newcommand{\Fpsi}[1]{\ensuremath{{\mathcal{F}^{#1}_{\textsf{psi}}}}\xspace}
%%%%%%%%%%%%%%%%%%%%

\newcommand{\etal}{{\sl et~al.}\xspace}
\newcommand{\eg}{{\sl e.g.}\xspace}
\newcommand{\ie}{{\sl i.e.}\xspace}
\newcommand{\apriori}{{\sl a~priori\/}\xspace}

%%%%%%%%%%%%%%%%%%%%%%

\newcommand{\rgets}{\overset{\$}\gets}

\newcommand{\command}[1]{\ensuremath{\text{\sc #1}}\xspace}

%%%%%%%%%%%%%%%%%%%%
\newcommand{\subname}[1]{\ensuremath{\textsc{#1}}\xspace}
\newcommand{\mycmd}[1]{\textsc{#1}}

\newcommand{\codebox}[1]{%
    \begin{varwidth}{\linewidth}%
    \upshape%   no slant in definition/theorem statement!
    \begin{tabbing}%
        \quad\=\quad\=\quad\=\quad\=\kill
        #1
    \end{tabbing}%
    \end{varwidth}%
}

\newcommand{\fcodebox}[1]{%
    \framebox{\codebox{#1}}%
}

\newcommand{\comment}[1]{
    \sl\small\color{black!50} \mbox{ // #1 }
}


%%%%%%%%%%%%%%%%%%%%


    \newtheorem{theorem}{Theorem}
    \newtheorem{definition}[theorem]{Definition}
    \newtheorem{claim}[theorem]{Claim}
    \newtheorem{lemma}[theorem]{Lemma}
    \newtheorem{proposition}[theorem]{Proposition}
\newtheorem{corol}[theorem]{Corollary}
\newtheorem{assumption}[theorem]{Assumption}
\newtheorem{obs}[theorem]{Observation}
\newtheorem{conj}[theorem]{Conjecture}

\newenvironment{proofof}[1]{\begin{proof}[Proof of #1.]}{\end{proof}}
\newenvironment{proofsketch}{\begin{proof}[Proof Sketch]}{\end{proof}}


%%%%%%%%%%%%%%%%%%%%%%%%%

\newcommand{\namedref}[2]{\hyperref[#2]{#1~\ref*{#2}}}
%% if you don't like it, use this instead:
%\newcommand{\namedref}[2]{#1~\ref{#2}}
\newcommand{\chapterref}[1]{\namedref{Chapter}{#1}}
\newcommand{\sectionref}[1]{\namedref{Section}{#1}}
\newcommand{\theoremref}[1]{\namedref{Theorem}{#1}}
\newcommand{\propositionref}[1]{\namedref{Proposition}{#1}}
\newcommand{\definitionref}[1]{\namedref{Definition}{#1}}
\newcommand{\corollaryref}[1]{\namedref{Corollary}{#1}}
\newcommand{\obsref}[1]{\namedref{Observation}{#1}}
\newcommand{\lemmaref}[1]{\namedref{Lemma}{#1}}
\newcommand{\claimref}[1]{\namedref{Claim}{#1}}
\newcommand{\figureref}[1]{\namedref{Figure}{#1}}
\newcommand{\tableref}[1]{\namedref{Table}{#1}}
\newcommand{\subfigureref}[2]{\hyperref[#1]{Figure~\ref*{#1}#2}}
\newcommand{\equationref}[1]{\namedref{Equation}{#1}}
\newcommand{\appendixref}[1]{\namedref{Appendix}{#1}}

\definecolor{darkred}{rgb}{0.5, 0, 0}
\definecolor{darkgreen}{rgb}{0, 0.5, 0}
\definecolor{darkblue}{rgb}{0, 0, 0.5}


%%%%%%%%%%%%%%%%%%%%%%%%%%%%%

\newcommand{\todo}[1]{%
    \mbox{}% prevent marginpar from being on previous paragraph
    \marginpar{%
        \colorbox{red!80!black}{\textcolor{white}{to-do}}%
        \vspace*{-22pt}% hack!
    }%
    \textcolor{red}{#1}%
}

\newcommand{\myparagraph}[1]{\smallskip\noindent{\bf #1:}}
\newcommand{\DB}{\ensuremath{DB}\xspace}
\newcommand{\encode}[1]{\ensuremath{\llbracket #1 \rrbracket}}
\newcommand{\strLen}{e}
\newcommand{\f}[1]{\ensuremath{\mathcal{F}_{\textsc{#1}}}}
\newcommand{\proto}[1]{\ensuremath{\Pi_{\textsc{#1}}}}
\newcommand{\numbins}{\ensuremath{\beta}\xspace}
\newcommand{\cuckoobins}{\ensuremath{m}\xspace}
\newcommand{\binsize}{\ensuremath{\mu}\xspace}
\newcommand{\symsec}{\ensuremath{\kappa}\xspace}
\newcommand{\statsec}{\ensuremath{\lambda}\xspace}
\newcommand{\clientsetsize}{\ensuremath{n}\xspace}
\newcommand{\serversetsize}{\ensuremath{N}\xspace}
\newcommand{\elsize}{\ensuremath{\rho}\xspace}
\newcommand{\binScale}{\ensuremath{c}\xspace}
\newcommand{\pirblocksize}{\ensuremath{b}\xspace}
\newcommand{\cuckooexpansion}{\ensuremath{e}\xspace}
\newcommand{\cuckootable}{\ensuremath{CT}\xspace}
